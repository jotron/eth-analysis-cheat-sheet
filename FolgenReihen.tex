\part{Folgen und Reihen}

\section{Konvergenz von Folgen}

% Definition 1
\Def Eine Folge $a_n$ \emph{konvergiert} gegen $a\in \R$, falls
\[
\forall \epsilon > 0 \ \exists N=N(\epsilon)\in\N, \ \forall n\geq
N \colon \abs{a_n-a} < \epsilon.
\]

% Definition 2
\Def Eine Folge $a_n$ \emph{konvergiert} gegen $a\in \R$, falls es $l \in \R$ gibt, so dass $\forall \epsilon > 0$ die Menge \\
$
\left\{n \in \mathbf{N}^{*}: a_{n} \notin\right] l-\varepsilon, l+\varepsilon[\}
$ endlich ist.

\sep

% Monotone Konvergenz
\Satz[Monotone] Sei $\left(a_{n}\right)_{n \geqslant 1}$ monoton fallend und nach unten beschränkt. Dann konvergiert
$\left(a_{n}\right)_{n \geqslant 1}$ mit Grenzwert
$
\lim _{n \rightarrow \infty} a_{n}=\inf \left\{a_{n}: n \geqslant 1\right\}
$.

% Cauchy
\Satz[Cauchy] Die Folge $\left(a_{n}\right)_{n \geqslant 1}$ ist genau dann konvergent, falls $\forall \varepsilon>0 \quad \exists N \geqslant 1 \quad$ so dass $\quad\left|a_{n}-a_{m}\right|<\varepsilon \quad \forall n, m \geqslant N$


\section{Konvergenz von Reihen}

% Cauchy
\Satz[Cauchy] Die Reihe $\sum_{k=1}^{\infty} a_{k}$ ist genau dann konvergent, falls. $\forall \varepsilon>0 \quad \exists N \geqslant 1 \quad$ mit $\quad\left|\sum\limits_{k=n}^{m} a_{k}\right|<\varepsilon \quad \forall m \geqslant n \geqslant N$

% Quotientenkriterum
\Satz[Ratio] Sei $\left(a_{n}\right)_{n \geqslant 1}$ mit $a_{n} \neq 0 \quad \forall n \geqslant 1 .$ Falls 
$$\limsup\limits_{n \rightarrow \infty} \frac{\left|a_{n+1}\right|}{\left|a_{n}\right|}<1$$ dann konvergiert die Reihe absolut.
Falls $\liminf\limits_{n \rightarrow \infty}\square > 1$ divergiert die Reihe.

% Wurzelkriterum
\Satz[Root] Falls $$\limsup\limits_{n \rightarrow \infty} \sqrt[n]{\left|a_{n}\right|}<1$$ dann konvergiert $\sum_{n=1}^{\infty} a_{n}$ absolut. Falls $\square > 1$, dann divergiert die Reihe.


% % % % %
% TRICKS
% % % % %
\section{Tricks}
\Lemma (Bernouilli) $(1+x)^{n} \geqslant 1+n \cdot x \quad \forall n \in \N, x>-1$.
