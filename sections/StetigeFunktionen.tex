\part{Stetige Funktionen}
\setcounter{section}{0}
% Allgemeine Stetigkeit
\Def Die Funktion $f: D \longrightarrow \R$ ist stetig falls sie in jedem Punkt von $D$ stetig ist.

% Gleichmässige stetigkeit
\Def Die Funktion $f: D \longrightarrow \R$ ist gleichmässig stetig, falls 
$\forall \epsilon > 0 \quad \exists\delta>0 \quad \forall x,y \in D$
$$|x-y|<\delta \Longrightarrow|f(x)-f(y)|<\varepsilon$$
. Insbesondere ist eine auf einem kompaktem Intervall stetige Funktion auch gleichmässig stetig.
% % % % %
% Stetigkeit an einem Punkt
% % % % %
\section{Stetigkeit an einem Punkt}

% Epsilon-Delta Definition
\Def[Epsilon] Sei $x_{0} \in D \subseteq \R$. Die Funktion $f: D \longrightarrow \R$ ist in $\boldsymbol{x}_{0}$ stetig, falls es für jedes $\varepsilon>0$ ein $\delta>0$ gibt, so dass für alle $x \in D$ gilt: $$\abs{x-x_{0}}<\delta \Longrightarrow\abs{f(x)-f\left(x_{0}\right)}<\varepsilon$$.

% Sequence Definition
\Satz[Sequence] Sei $x_{0} \in D \subseteq \R$. Die Funktion $f: D \longrightarrow \R$ ist genau dann in $\boldsymbol{x}_{0}$ stetig, falls für jede Folge $\left(a_{n}\right)_{n \geqslant 1}$ in $D$
	$$\lim \limits_{n \rightarrow \infty} a_{n}=x_{0} \Longrightarrow \lim \limits_{n \rightarrow \infty} f\left(a_{n}\right)=f\left(x_{0}\right)$$ gilt.
	
% Limit
\Satz[Sidewise] Sei $x_{0} \in D \subseteq \R$. Die Funktion $f: D \longrightarrow \R$ ist in $\boldsymbol{x}_{0}$ stetig, falls 
	$$f(x_0) =\lim \limits_{x \rightarrow x_0} f(x) =  \lim \limits_{x \rightarrow x_0^+} f(x) = \lim \limits_{x \rightarrow x_0^-} f(x)$$ gilt.	
	
% Differnzierbarkeit
\Satz[Differentiable] Sei $x_{0} \in D \subseteq \R$. Die Funktion $f: D \longrightarrow \R$ ist in $\boldsymbol{x}_{0}$ stetig, falls sie $\boldsymbol{x}_{0}$ differenzierbar ist.

% % % % %
% Eigenschaften an einem Punkt
% % % % %
\section{Eigenschaften}

\Satz[Zwischenwertsatz] Sei $I \subset \R$ ein Intervall, $f: I \longrightarrow \R$ eine stetige Funktion und $a, b, \in I$. Für jedes $y$ zwischen $f(a)$ und $f(b)$ gibt es ein $x$ zwischen $a$ und $b$ mit $f(x)=y$.

\Satz[Min-Max] Sei $f: I=[a, b] \longrightarrow \R$ stetig auf einem kompakten Intervall $I$. Dann gibt es $u \in I$ und $v \in I$ mit
$$f(u) \leqslant f(x) \leqslant f(v) \quad \forall x \in I$$
und f ist beschränkt.

\Satz[Umkehrabbildung] Sei $I \subset \R$ ein Intervall und $f: I \longrightarrow \R$ stetig, streng monoton. Dann ist $J:=f(I) \subset \R$ ein Intervall und $f^{-1}: J \longrightarrow I$ ist stetig, streng monoton.

% % % % %
% Funktionenfolgen
% % % % %
\section{Konvergenz von Funktionenfolgen}

\Def[Punktweise] Eine Folge stetiger Funktionen $f_n\colon\Omega\subset\R\to\R$
konvergiert punktweise gegen $f(x)$, falls
\[
\forall x\in\Omega \ \lim_{n\to\infty} f_n(x) = f(x).
\]

\Def[Gleichmässig] Eine Folge stetiger Funktionen $f_n\colon\Omega\subset\R\to\R$
konvergiert gleichmässig gegen $f$, falls
\[
\lim_{n\to\infty} \sup_{x\in\Omega} \abs{f_n(x)-f(x)} = 0.
\]
bzw. falls gilt: $\forall \varepsilon>0  \quad \exists N \geqslant 1,$ so dass:
$$
\forall n \geqslant N, \quad \forall x \in D: \quad\left|f_{n}(x)-f(x)\right|<\varepsilon
$$

\Satz[Stetige Funktionenfolge] Sei $D \subset R$ und $f_{n}: D \rightarrow \R$ eine Funktionenfolge bestehend aus (in
D) stetigen Funktionen die (in D) gleichmässig gegen eine Funktion $f: D \rightarrow \R$ konvergieren. Dann ist $f$ (in D) stetig.

\Satz[Beschränkte Funktionenfolge] Sei $D \subset R$ und $f_{n}: D \rightarrow \R$ eine Folge stetiger Funktionen. Falls  $\left|f_{n}(x)\right| \leqslant c_{n} \quad \forall x \in D$ und $\sum_{n=0}^{\infty} c_{n}$ konvergiert dann konvergiert 
$$\sum\limits_{n=0}^{\infty} f_n(x)  =: f(x)$$
ebenfalls und deren Grenzwert $f$ ist eine in $D$ stetige Funktion.

% % % % %
% Grenzwert an einem Punkt
% % % % %
\section{Grenzwert an einem Punkt}

\Def[Häufungspunkt] $x_{0} \in \R$ ist ein Häufungs -punkt der Menge $D$ falls $\forall \delta>0$ gilt:
$$(]x_{0}-\delta, x_{0}+\delta\left[\backslash\left\{x_{0}\right\}\right) \cap D \neq \varnothing$$

\Def[Grenzwert] $\lim \limits_{x \rightarrow x_0} f(x) = A$ 
mit $A \in \R$, $f: D \longrightarrow \R$, falls $x_0 \in \R$ ein Häufungspunkt ist und $\forall \varepsilon>0 \quad \exists \delta>0$
$$\forall x \in D \cap(] x_{0}-\delta, x_{0}+\delta\left[\backslash\left\{x_{0}\right\}\right):|f(x)-A|<\varepsilon$$


