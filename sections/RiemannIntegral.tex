\part{Riemann Integral}
\setcounter{section}{0}


% % % % %
% Integrationskriterien
% % % % %
\section{Integrationskriterien}

% Riemann Summe
\Def Sei $f : [a,b] \longrightarrow \R$, $P$ eine Partition ($P \subset [a,b]$ und $\{a,b\} \subset P$), $\delta_{i} = x_i-x_{i-1}$,
und $\mathcal{P}(I)$ die Menge der Partitionen, wir definieren die Untersummen:
$$s(f, P):=\sum_{i=1}^{n} f_{i} \delta_{i} \quad, f_i = \inf_{x_{i-1}\leq x \leq x_i} f(x)$$
$$s(f):=\sup_{P \in \mathcal{P}(I)} s(f, P)$$
und die Obersummen:
$$S(f, P):=\sum_{i=1}^{n} F_{i} \delta_{i} \quad, F_i = \sup_{x_{i-1}\leq x \leq x_i} f(x)$$
$$S(f):=\inf_{P \in \mathcal{P}(I)} S(f, P)$$

% Integrierbarkeit
\Def Eine beschränkte Funktion $f : [a,b] \longrightarrow \R$ ist integrierbar falls
$$s(f) = S(f) \quad := \int_{a}^{b} f(x) dx$$

% Satz zur Integrierbarkeit
\Satz Eine beschränkte Funktion $f : [a,b] \longrightarrow \R$ ist integrierbar falls
$$\forall \varepsilon>0 \quad \exists P \in \mathcal{P}(I) \quad \text { mit } \quad S(f, P)-s(f, P)<\varepsilon$$

% Satz zur Integrierbarkeit
\Satz $f : [a,b] \longrightarrow \R$  stetig $\Rightarrow$ integrierbar.

% Satz zur Integrierbarkeit
\Satz $f : [a,b] \longrightarrow \R$  monoton $\Rightarrow$ integrierbar.

\Satz Seien $f,g : [a,b] \longrightarrow \R$ beschränkt integrierbar und $\lambda \in \R$, dann sind $f+g$, $\lambda \cdot f$, $f \cdot g$, 
$\max (f,g)$, $\min (f,g)$, $|f|$,  $f / g$ (falls $g(x) \geq \beta > 0 \quad \forall x$) integrierbar.

% NOTE: LEFTOUT: Kor. 5.1.9
% NOTE: LEFTOUT: Satz. 5.2.10

\section{Eigenschaften}

\Satz[Cauchy-Schwarz] Seien $f,g : [a,b] \longrightarrow \R$ beschränkt integrierbar, dann gilt
$$
\left|\int_{a}^{b} f(x) g(x) d x\right| \leqslant \sqrt{\int_{a}^{b} f^{2}(x) d x} \sqrt{\int_{a}^{b} g^{2}(x) d x}
$$

\Satz[Mittelwertsatz] Seien $f : [a,b] \longrightarrow \R$ stetig, dann $\exists \xi \in [a,b]$ mit:
$$\int_a^b f(x) dx = f(\xi)(b-a)$$

% NOTE: LEFTOUT: Satz. 5.3.6

\Satz[Stammfunktion] Seien $a < b$ und $f : [a,b] \longrightarrow \R$. Eine Funktion $F : [a,b] \longrightarrow \R$  heisst Stammfunktion von $f$, falls $F$ (stetig) differenzierbar in $[a,b]$ ist und $F' = f$ in $[a,b]$ gilt. \\
$$F(x) = \int_a^x f(t) dt$$ ist eine Stammfunktion von $f$.

\Satz[Fundamentalsatz] Sei $f : [a,b] \longrightarrow \R$ stetig, dann gilt
$$\int_a^b f(x) dx = F(b)-F(a)$$

\Satz[Partielle Int.] Seien $a < b$ reelle Zahlen und $f : [a,b] \longrightarrow \R$ stetig differenzierbar
$$
\int_{a}^{b} f(x) g^{\prime}(x) dx = \Big[f(x)g(x)\Big]_a^b - \int_{a}^{b} f^{\prime}(x) g(x) dx
$$

\Satz[Substitution] Sei $a<b$, $\phi : [a,b] \longrightarrow \R$ stetig differenzierbar, $I \subset \R$ ein Intervall mit $\phi([a,b]) \subset I$ und $f: I \longrightarrow \R$ eine stetige Funktion. Dann gilt:
$$
\int_{\phi(a)}^{\phi(b)} f(x) dx=\int_{a}^{b} f(\phi(t)) \phi^{\prime}(t) dt
$$

\Satz Sei $f_n : [a,b] \longrightarrow \R$ eine Folge von beschränkten, integrierbaren Funktionen die gleichmässig konvergieren, dann gilt
$$
\lim_{n \rightarrow \infty} \int_a^b f_n(x) dx = \int_a^b \lim_{n \rightarrow \infty} f_n(x) dx
$$

\Satz[Stirling] 
$$
n !=\frac{\sqrt{2 \pi n} n^{n}}{e^{n}} \cdot \exp \left(\frac{1}{12 n}+R_{3}(n)\right)
$$
$$
\left|R_{3}(n)\right| \leqslant \frac{\sqrt{3}}{216} \cdot \frac{1}{n^{2}} \quad \forall n \geqslant 1
$$

% % % % %
% Uneigentliche Integrale
% % % % %
\section{Uneigentliche Integrale}

\Def Sei $f:[a,\infty] \longrightarrow \R$ beschränkt und integrierbar auf $[a,b] \quad \forall b \geq a$, wir definieren
$$\int_a^\infty f(x) dx := \lim_{b \rightarrow \infty} \int_a^b f(x) dx$$
und falls $f$ auf $[a+\epsilon,b], \epsilon>0$ beschränkt und integrierbar ist, aber nicht beschränkt auf $]a,b]$, dann
$$\int_a^b f(x) dx := \lim_{\epsilon \rightarrow 0} \int_{a+\epsilon}^b f(x) dx$$

\Lemma Sei $f:[a,\infty] \longrightarrow \R$ beschränkt und integrierbar auf $[a,b] \quad \forall b \geq a$.
\begin{enumerate}
  \item Falls $|f(x)| \leqslant g(x) \quad \forall x \geqslant a$ und $g(x)$ ist auf $[a, \infty[$ integrierbar, so ist f auf $[a, \infty[$ integrierbar.
  \item Falls $0 \leqslant g(x) \leqslant f(x)$ und $\int_a^\infty g(x) dx$ divergiert, so divergiert auch $\int_a^\infty f(x) dx$
\end{enumerate}

% % % % %
% Partialbruchzerlegung
% % % % %
\section{Partialbruchzerlegung}
\Trick Sei $R(x) = \frac{P(x)}{Q(x)}$ eine rationale Funktion und $\text{grad}(P) < \text{grad}(Q)$, dann ist
\begin{align*}
Q(x) &= x^n + a_{n-1}x'{n-1}+\dots \\
	 &=	\prod_{i=1}^{k}\left(x-\gamma_{i}\right)^{n_{i}} \prod_{j=1}^{l}\left(\left(x-\alpha_{j}\right)^{2}+\beta_{j}^{2}\right)^{m_{j}} 
\end{align*}
und
\begin{align*}
\frac{P(x)}{Q(x)}=&\sum_{i=1}^{k} \sum_{j=1}^{n_{i}} \frac{C_{i j}}{\left(x-\gamma_{i}\right)^{j}}+ \\
				 &\sum_{i=1}^{l} \sum_{j=1}^{m_{i}} \frac{\left(A_{i j}+B_{i j} x\right)}{\left(\left(x-\alpha_{i}\right)^{2}+\beta_{i}^{2}\right)^{j}}
\end{align*}

% % % % %
% Wichtige Beispiele
% % % % %
\section{Wichtige Beispiele}
\Bsp[Gamma Funktion]
$$
\Gamma(s):=\int_{0}^{\infty} e^{-x} x^{s-1} d x
$$
Weitere Eigenschaften sind:
\begin{itemize}
	\item $\Gamma(1) = 1$
	\item $\Gamma(s+1) = s \Gamma(s) \forall s>0$
	\item $\Gamma(n+1) = n!$
	\item logarithmisch konvex
	\item $\Gamma(x)=\lim _{n \rightarrow+\infty} \frac{n ! n^{x}}{x(x+1) \cdots(x+n)} \quad \forall x>0$
\end{itemize}
