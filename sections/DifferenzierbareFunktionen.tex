\part{Differenzierbare Funktionen}
\setcounter{section}{0}
% Allgemeine Diff
\Def Die Funktion $f: D \longrightarrow \R$ ist differenzierbar falls sie in jedem Punkt von $D$ differenzierbar ist.

% % % % %
% diff. an einem Punkt
% % % % %
\section{Differenzierbarkeit}

\Def $f$ ist in $x_0$ differenzierbar falls 
$$\lim \limits_{x \rightarrow x_{0}} \frac{f(x)-f\left(x_{0}\right)}{x-x_{0}}$$
existiert. Falls $x=x_0+h$, ist dies äquivalent zu
$$\lim \limits_{h \rightarrow 0} \frac{f\left(x_{0}+h\right)-f\left(x_{0}\right)}{h}$$

\Satz $f : D \longrightarrow \R$ ist in genau dann in $x_0$ differenzierbar falls es eine in $x_0$ stetige Funktion $\phi : D \longrightarrow \R$ gibt, so dass
$$f(x) = f(x_0) + \phi(x)(x-x_0) \quad \forall x \in D$$


\section{Ableitungen}
\Satz[Ableitungsregeln]
\begin{itemize}
  \item Summenregel
  \[(f+g)'(x_0) = f'(x_0) + g'(x_0)\]
  \item Produktregel
  \[(f\cdot g)'(x_0) = f'(x_0)g(x_0) + f(x_0)g'(x_0)\]
  \item Quotientenregel
  \[\left(\frac{f}{g}\right)'=
  \frac{f'(x_0)g(x_0) - f(x_0)g'(x_0)}{g^2(x_0)}\]
  \item Kettenregel \[
	(g\circ f)'(x_0) = g'(f(x_0))\cdot f'(x_0).
  \]
\end{itemize}

\Korollar[Inverse] Sei $f : D \longrightarrow E$ eine bijektive Funktion, f in $x_0$ differenzierbar, $f'(x_0) \neq 0$ und $f^{-1}$ ist in $y_0 = f(x_0)$ stetig, dann gilt
$$(f^{-1})'(y_0) = \frac{1}{f'(x_0)}$$

\section{Zentrale Sätze}
\Satz[Extrema] $f : \R \longrightarrow \R$ besitzt ein lokales Max./Min. in $x_0$ falls es $\delta >0$ gibt mit:
		$$f(x) \lessgtr f(x_0) \quad \forall x \in ]x_0-\delta, x_0+\delta[ $$
In beiden Fällen gilt $f'(x_0) = 0$.

\Satz Falls $f'(x_0) \lessgtr 0$ gibt es $\delta > 0$ mit
\begin{align*}
	&f(x) \lessgtr f(x_0) \quad &\forall x \in ]x_0, x_0+\delta[ \\
	&f(x) \lessgtr f(x_0) \quad &\forall x \in ]x_0-\delta, x_0[ 
\end{align*}

\Satz[Lagrange/Mean] Sei $f:[a,b] \longrightarrow \R$ stetig und in $]a,b[$ diff., dann gibt es $\xi \in ]a,b[$ mit
$$f(b)-f(a) = f'(\xi)(b-a)$$

\Satz[L'Hospital] Seien $f,g: ]a,b[ \longrightarrow \R$ diff. mit $g'(x) \neq 0 \forall x \in [a,b]$. Falls
$$\lim _{x \rightarrow b^{-}} f(x)=0, \quad \lim _{x \rightarrow b^{-}} g(x)=0, \quad \lim _{x \rightarrow b^{-}} \frac{f'(x)}{g'(x)}=:\lambda$$
dann folgt
$$\lim _{x \rightarrow b^{-}} \frac{f(x)}{g(x)}=\lambda$$


\Def[Konvex] $f$ ist konvex (auf I) falls $\forall x \leq y$ und $\lambda \in [0,1]$ gilt
$$f(\lambda x+(1-\lambda) y) \leqslant \lambda f(x)+(1-\lambda) f(y)$$

\Lemma[Konvex] Seie $f: ]a,b[ \longrightarrow \R$ und $f \in C^2$ 
$$ f \text{ konvex} \iff \forall x \quad f''(x) \geq 0$$

\Def[Glatt] Die Funktion $f$ ist glatt falls sie $\forall n \geq 1$, n-mal differenzierbar ist.

\Satz[Funktionenfolgen] Sei $f_n:]a,b[ \longrightarrow \R$ eine Funktionenfolge wobei $f_n \forall n$ einmal stetig differenzierbar ist. Falls $(f_n)_{n\geqslant1}$ und $(f_n')_{n\geqslant1}$ gleichässig in $]a,b[$ konvergieren gilt:
$$(\lim _{n \rightarrow \infty} f_{n})'=\lim _{n \rightarrow \infty} f_{n}^{\prime}$$

\Satz[Taylor Approximation] Sei $f:[a,b] \longrightarrow \R$ stetig und in $]a,b[$ (n+1)-mal diff.. Für jedes $a<x\leqslant b$ gibt es $\xi \in ]a,x[$ mit:
$$f(x)=\sum_{k=0}^{n} \frac{f^{(k)}(a)}{k !}(x-a)^{k}+\frac{f^{(n+1)}(\xi)}{(n+1) !}(x-a)^{n+1}$$


% % % % %
% Wichtige Beispiele
% % % % %
\section{Wichtige Beispiele}
\Bsp[Exponentialfunktion]
$$\exp (z):=\sum_{n=0}^{\infty} \frac{z^{n}}{n !} \quad \exp (z)' = \exp(z)$$
$\exp : \R \longrightarrow] 0,+\infty[$ ist streng monoton wachsend, differenzierbar, und surjektiv. Beobachte dass $\exp (x) \geqslant 1+x \quad \forall x \in \R$.
Die Umkehrabbildung ist 
$$\ln :] 0,+\infty[\longrightarrow \R \quad \ln (x)' = 1/x$$
wobei $\ln$ eine streng monoton wachsende, differenzierbare, bijektive Funktion ist. \\
\Lemma $e^x > 1+x$

\sep
\Bsp[Trigonometrische Funkt.]
\begin{align*}
\sin(\varphi)  
& =\sum_{k=0}^{\infty} (-1)^k \frac{\varphi^{2k+1}}{(2k+1)!} &\sin(\varphi)' = \cos(\varphi)  \\
\cos(\varphi)  
& = \sum_{k=0}^{\infty} (-1)^k \frac{\varphi^{2k}}{(2k)!} &\cos(\varphi)' = -\sin(\varphi) \\
\tan(\varphi)  
& = \frac{\sin(\varphi)}{\cos(\varphi)} & \tan(\varphi)' = \frac{1}{\cos(\varphi)^2} 
\end{align*}
Merke dass $\int \tan(x) = -\ln(|\cos(x)|)$.

\Korollar 
\begin{align*}
\forall \varphi > 0 \quad \exists \tau \in[0, \varphi] \quad \sin(\varphi) &= \varphi - \frac{\varphi^3}{6}\cos(\tau)
\end{align*}

\Satz $\forall z \in \C$ 
\begin{itemize}
	\item $\exp(iz) = \cos(z) + i\sin(z)$
	\item $\cos(z)^2 + \sin(z)^2 = 1$
	\item $\sin(z+w) = sin(z)cos(w) + sin(w)cos(z)$ \\
		  $\cos(z+w) = cos(z)cos(w) - sin(w)sin(z)$ 
	\item $\sin(z) = \frac{e^{iz}-e^{-iz}}{2i}$, $\cos(z) = \frac{e^{iz}+e^{-iz}}{2}$
\end{itemize}
\Lemma
\begin{align*}
\arcsin(y)'  &= \frac{1}{\sqrt{1-y^2}}  &[-1,1] \longrightarrow [-\pi/2, \pi/2]  \\
\arccos(y)'  &= \frac{-1}{\sqrt{1-y^2}}  &[-1,1] \longrightarrow [0, \pi]  \\
\arctan(y)'  &= \frac{1}{1+y^2}  &[-\infty,\infty] \longrightarrow [-\pi/2, \pi/2]  
\end{align*}

\Lemma
\begin{align*}
\sin(\arccos(x))  &= \cos(\arcsin(x)) &= \sqrt{1-x^2} \\
\arcsin(\cos(x)) &= x + \pi/2 
\end{align*}

\sep

\Bsp[Hyperbolische Funkt.]
\begin{align*}
\sinh(x)  & =\frac{e^{x}-e^{-x}}{2} &\arcsinh(y)' = \frac{1}{\sqrt{1+y^2}}  \\
\cosh(x)  & =\frac{e^{x}+e^{-x}}{2} &\arccosh(y)' =  \frac{1}{\sqrt{y^2-1}} \\
\tanh(x)  & =\frac{e^{x}-e^{-x}}{e^{x}+e^{-x}} & \arctanh(y)' = \frac{1}{1-y^2}
\end{align*}
wobei arsinh $ : \R \rightarrow \R$, arcosh $ : [1,\infty[ \rightarrow [0,\infty[$ und artanh $: ]-1,1[ \rightarrow \R$.
\begin{itemize}
	\item $\cosh(z)^2 - \sinh(z)^2 = 1$
	\item $\sinh(x)' = \cosh(x)$
	\item $\cosh(x)' = \sinh(x)$
\end{itemize}
\sep

\begin{table}[H]
\centering
\begin{tabular}{p{2cm}p{4cm}}
$\int \ln(x) dx$ & $=x\ln(x)-x+C$ \\
$\int x e^x dx$ & $=x e^x-x+C$ \\
$\int x \cos(x) dx$ & $=x\sin(x)+\cos(x)+C$ \\
$\int x \sin(x) dx$ & $=sin(x)-x\cos(x)+C$ \\
$\int \sin(x)^2 dx$ & $=(2x-sin(2x))/4+C$ \\
$\int \cos(x)^2 dx$ & $=(\cos(x)\sin(x)+x)/2+C$ \\
\end{tabular}
\end{table}
