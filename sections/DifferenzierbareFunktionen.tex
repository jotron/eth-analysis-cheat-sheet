\part{Differenzierbare Funktionen}
\setcounter{section}{0}
% Allgemeine Diff
\Def Die Funktion $f: D \longrightarrow \R$ ist differenzierbar falls sie in jedem Punkt von $D$ differenzierbar ist.

% % % % %
% diff. an einem Punkt
% % % % %
\section{Differenzierbarkeit}

\Def $f$ ist in $x_0$ differenzierbar falls 
$$\lim \limits_{x \rightarrow x_{0}} \frac{f(x)-f\left(x_{0}\right)}{x-x_{0}}$$
existiert. Falls $x=x_0+h$, ist dies äquivalent zu
$$\lim \limits_{h \rightarrow 0} \frac{f\left(x_{0}+h\right)-f\left(x_{0}\right)}{h}$$

\section{Abbleitungen}
\Satz[Ableitungsregeln]
\begin{itemize}
  \item Summenregel
  \[(f+g)'(x_0) = f'(x_0) + g'(x_0)\]
  \item Produktregel
  \[(f\cdot g)'(x_0) = f'(x_0)g(x_0) + f(x_0)g'(x_0)\]
  \item Quotientenregel
  \[\left(\frac{f}{g}\right)'=
  \frac{f'(x_0)g(x_0) - f(x_0)g'(x_0)}{g^2(x_0)}\]
  \item Kettenregel \[
	(g\circ f)'(x_0) = g'(f(x_0))\cdot f'(x_0).
  \]
\end{itemize}

% L17
% TODO exp(x)', sin(x)', cos(x)', tan(x)'
% TODO (f inverse)'

% L18
% TODO f'(x)=0 in local extrema

% Eigenschaften:
% TODO f(a) = f(b), es gibt a < x < b mit f'(x) = 0.
% Mittelwertsatz, zwischen zwei stellen a und b, muss es einen punkt geben mit f'(x) = f(a)-f(b) /a-b, sonst wäre die steigung konstant kleiner oder grösser was ein widerspruch wäre.

