\part{Differenzierbare Funktionen}
\setcounter{section}{0}
% Allgemeine Diff
\Def Die Funktion $f: D \longrightarrow \R$ ist differenzierbar falls sie in jedem Punkt von $D$ differenzierbar ist.

% % % % %
% diff. an einem Punkt
% % % % %
\section{Differenzierbarkeit}

\Def $f$ ist in $x_0$ differenzierbar falls 
$$\lim \limits_{x \rightarrow x_{0}} \frac{f(x)-f\left(x_{0}\right)}{x-x_{0}}$$
existiert. Falls $x=x_0+h$, ist dies äquivalent zu
$$\lim \limits_{h \rightarrow 0} \frac{f\left(x_{0}+h\right)-f\left(x_{0}\right)}{h}$$

\section{Ableitungen}
\Satz[Ableitungsregeln]
\begin{itemize}
  \item Summenregel
  \[(f+g)'(x_0) = f'(x_0) + g'(x_0)\]
  \item Produktregel
  \[(f\cdot g)'(x_0) = f'(x_0)g(x_0) + f(x_0)g'(x_0)\]
  \item Quotientenregel
  \[\left(\frac{f}{g}\right)'=
  \frac{f'(x_0)g(x_0) - f(x_0)g'(x_0)}{g^2(x_0)}\]
  \item Kettenregel \[
	(g\circ f)'(x_0) = g'(f(x_0))\cdot f'(x_0).
  \]
\end{itemize}

% TODO (f inverse)'

% L18
% TODO f'(x)=0 in local extrema

% Eigenschaften:
% TODO f(a) = f(b), es gibt a < x < b mit f'(x) = 0.
% Mittelwertsatz, zwischen zwei stellen a und b, muss es einen punkt geben mit f'(x) = f(a)-f(b) /a-b, sonst wäre die steigung konstant kleiner oder grösser was ein widerspruch wäre.


% % % % %
% Wichtige Beispiele
% % % % %
\section{Wichtige Beispiele}
\subsubsection{Exponentialfunktion}
$$\exp (z):=\sum_{n=0}^{\infty} \frac{z^{n}}{n !} \quad \exp (z)' = \exp(z)$$
$\exp : \R \longrightarrow] 0,+\infty[$ ist streng monoton wachsend, differenzierbar, und surjektiv. Beobachte dass $\exp (x) \geqslant 1+x \quad \forall x \in \R$.
Die Umkehrabbildung ist 
$$\ln :] 0,+\infty[\longrightarrow \R \quad \ln (x)' = 1/x$$
wobei $\ln$ eine streng monoton wachsende, differenzierbare, bijektive Funktion ist.
\sep
\subsubsection{Trigonometrische Funkt.}
\begin{align*}
\sin(\varphi)  
& =\sum_{k=0}^{\infty} (-1)^k \frac{\varphi^{2k+1}}{(2k+1)!} &\sin(\varphi)' = \cos(\varphi)  \\
\cos(\varphi)  
& = \sum_{k=0}^{\infty} (-1)^k \frac{\varphi^{2k}}{(2k)!} &\cos(\varphi)' = -\sin(\varphi) \\
\tan(\varphi)  
& = \frac{\sin(\varphi)}{\cos(\varphi)} & \tan(\varphi)' = \frac{1}{\cos(\varphi)^2}
\end{align*}

\Korollar 
\begin{align*}
\forall \varphi > 0 \quad \exists \tau \in[0, \varphi] \quad \sin(\varphi) &= \varphi - \frac{\varphi^3}{6}\cos(\tau)
\end{align*}

\Satz $\forall z \in \C$ 
\begin{itemize}
	\item $\exp(iz) = \cos(z) + i\sin(z)$
	\item $\cos(z)^2 + \sin(z)^2 = 1$
	\item $\sin(z+w) = sin(z)cos(w) + sin(w)cos(z)$ \\
		  $\cos(z+w) = cos(z)cos(w) - sin(w)sin(z)$ 
	\item $\sin(z) = \frac{e^{iz}-e^{-iz}}{2i}$, $\cos(z) = \frac{e^{iz}+e^{-iz}}{2}$
\end{itemize}
\sep
\subsubsection{Hyperbolische Funkt.}
\todo