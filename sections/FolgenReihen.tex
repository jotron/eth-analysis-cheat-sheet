\part{Folgen und Reihen}

\section{Konvergenz von Folgen}

% Definition 1
\Def[1] Eine Folge $a_n$ \emph{konvergiert} gegen $a\in \R$, falls
\[
\forall \epsilon > 0 \ \exists N=N(\epsilon)\in\N, \ \forall n\geq
N \colon \abs{a_n-a} < \epsilon.
\]
Für $\R^d$ muss gelten $\norm{a_n-a} < \epsilon$.

% Definition 2
\Def[2] Eine Folge $a_n$ \emph{konvergiert} gegen $a\in \R$, falls es $l \in \R$ gibt, so dass $\forall \epsilon > 0$ die Menge \\
$
\left\{n \in \mathbf{N}^{*}: a_{n} \notin\right] l-\varepsilon, l+\varepsilon[\}
$ endlich ist.

\sep

% Monotone Konvergenz
\Satz[Monotone] Sei $\left(a_{n}\right)_{n \geqslant 1}$ monoton fallend und nach unten beschränkt. Dann konvergiert
$\left(a_{n}\right)_{n \geqslant 1}$ mit Grenzwert
$
\lim _{n \rightarrow \infty} a_{n}=\inf \left\{a_{n}: n \geqslant 1\right\}
$.

% Cauchy
\Satz[Cauchy] Die Folge $\left(a_{n}\right)_{n \geqslant 1}$ ist genau dann konvergent, falls $\forall \varepsilon>0 \quad \exists N \geqslant 1 \quad$ so dass $\quad\left|a_{n}-a_{m}\right|<\varepsilon \quad \forall n, m \geqslant N$

% Sandwich
\Satz[Sandwich] Die Folge $\left(a_{n}\right)_{n \geqslant 1}$ konvergiert zu $a$, falls $\left(b_{n}\right)_{n \geqslant 1}, \left(c_{n}\right)_{n \geqslant 1}$ existieren mit Grenzwert $a$ und $\forall n\geq1 : b_n \leq a_n \leq c_n$.


\section{Konvergenz von Reihen}

% Absolute Convergence
\Def Die Reihe $\sum_{k=1}^{\infty} a_{k}$ konvergiert absolut ($\Rightarrow$ konvergent), falls $\sum\limits_{k=1}^{\infty} \abs{a_{k}}$ kovergiert.

% Cauchy
\Satz[Cauchy] Die Reihe $\sum_{k=1}^{\infty} a_{k}$ ist genau dann konvergent, falls. $\forall \varepsilon>0 \quad \exists N \geqslant 1 \quad$ mit $\quad\left|\sum\limits_{k=n}^{m} a_{k}\right|<\varepsilon \quad \forall m \geqslant n \geqslant N$

% Quotientenkriterum
\Satz[Ratio] Sei $\left(a_{n}\right)_{n \geqslant 1}$ mit $a_{n} \neq 0 \quad \forall n \geqslant 1 .$ Falls 
$$\limsup\limits_{n \rightarrow \infty} \frac{\left|a_{n+1}\right|}{\left|a_{n}\right|}<1$$ dann konvergiert die Reihe absolut.
Falls $\liminf\limits_{n \rightarrow \infty}\square > 1$ divergiert die Reihe.

% Wurzelkriterum
\Satz[Root] Falls $$\limsup\limits_{n \rightarrow \infty} \sqrt[n]{\left|a_{n}\right|}<1$$ dann konvergiert $\sum_{n=1}^{\infty} a_{n}$ absolut. Falls $\square > 1$, dann divergiert die Reihe.

% Alternierende Reihe
\Satz[Alternating] Sei $\left(a_{n}\right)_{n \geqslant 1}$ monoton fallend mit $a_{n} \geqslant 0 \quad \forall n \geqslant 1$ und $\lim \limits_{n \rightarrow \infty} a_{n}=0 .$ Dann konvergiert 
$$S:=\sum_{k=1}^{\infty}(-1)^{k+1} a_{k}$$ und es gilt $a_{1}-a_{2} \leqslant S \leqslant a_{1}$.

\Bsp $\sum_{k=1}^{\infty}(-1)^{k+1} \frac{1}{k}$ konvergiert zu $\log(2)$.

\Bsp $\sum_{k=1}^{\infty}(-1)^{k+1} \frac{x^k}{k}$ konvergiert zu $\log(1+x)$ für $-1<x<1$.

% Integral Hilfe
\Satz[McLaurin] Sei $f:[1,\infty[ \longrightarrow [0, \infty[$ monoton fallend.
$$ \sum_{n=1}^\infty f(n) \text{ konvergiert} \Longleftrightarrow \int_1^\infty f(x) dx \text{ konv.}$$
und in diesem Fall gilt \\
$ 0 \leq \sum_{n=1}^\infty f(n) - \int_1^\infty f(x) dx \leq f(1)$

\Korollar[Vergleichssatz] Seien $\sum_{k=1}^{\infty} a_{k}$ und $\sum_{k=1}^{\infty} b_{k}$ Reihen mit: $0 \leq a_{k} \leq b_{k} \quad \forall k \geq 1. $
\[ \sum_{k=1}^{\infty} b_{k} \text{ konvergent} \implies \sum_{k=1}^{\infty} a_{k} \text{ konvergent} \]
\[ \sum_{k=1}^{\infty} a_{k} \text{ divergent} \implies \sum_{k=1}^{\infty} b_{k} \text{ divergent} \]


% % % % %
% TRICKS
% % % % %
\section{Eigenschaften}
\Lemma (Bernouilli) $(1+x)^{n} \geqslant 1+n \cdot x \quad \forall n \in \N, x>-1$.

% Bolzano Weierstrass
\Satz[Teilfolge] Jede beschränkte Folge besitzt eine konvergente Teilfolge.

% R^d
\Satz[Vektorfolge] $\lim \limits_{n \rightarrow \infty} a_{n}=b$ genau dann wenn $\lim \limits_{n \rightarrow \infty} a_{n, j}=b_{j} \quad \forall 1 \leqslant j \leqslant d$.

% Lim Sup, Lim Inf
\Def[LimSup, LimInf] Sei $a_n$ beschränkt, definieren wir
$$\limsup\limits_{n \rightarrow \infty} a_n := \lim \limits_{n \rightarrow \infty} \sup \{a_k : k\geqslant n\}$$
$$\liminf\limits_{n \rightarrow \infty} a_n := \lim \limits_{n \rightarrow \infty} \inf \{a_k : k\geqslant n\}$$

\sep
% Dirichlet, Umordnungen
\Satz[Umordnung] Falls eine Reihe absolut konvergiert, dann konvergiert jede
Umordnung der Reihe und hat denselben Grenzwert.

\Satz[2.7.23] Falls $\sum_{i=0}^{m}\sum_{j=0}^{m}\abs{a_{ij}} \leq B, \quad \forall m \geq 0$
\[ \text{dann konvergiert } S_{i} := \sum_{j=0}^{\infty} a_{ij} \quad \forall i \geq 0 \]
\[ \text{dann konvergiert } U_{j} := \sum_{i=0}^{\infty} a_{ij} \quad \forall j \geq 0 \]
\[ \text{und es gilt } \sum_{i=0}^{m} S_{i} = \sum_{j=0}^{m} U_{j} \]

\Satz[2.7.24] Das \textbf{Cauchy Produkt} der Reihen $\sum_{i=0}^{\infty} a_i, \ \sum_{i=0}^{\infty} b_i$ ist die Reihe
\[\sum_{n=0}^\infty \Bigg(\sum_{j=0}^{n} a_{n-j} b_{j} \Bigg) = a_0 b_0 + (a_0 b_1 + a_1 b_0) + \cdots  \]

\Satz[2.7.26] Falls die Reihen $\sum_{i=0}^{\infty} a_i, \ \sum_{i=0}^{\infty} b_i$ absolut konvergieren, so knovergiert ihr Cauchy Produkt und es gilt:
\[\sum_{n=0}^\infty \Bigg(\sum_{j=0}^{n} a_{n-j} b_{j} \Bigg) = \Bigg( \sum_{i=0}^\infty a_i \Bigg) \Bigg(\sum_{j=0}^\infty b_j \Bigg) \]

% % % % %
% Bekannte Grenzwerte
% % % % %


\section{Wichtige Beispiele}
\Bsp[Geometrische Reihe] Für $q<1$, konvergiert $\sum_{n=0}^N q^n$ zu $\frac{1-q^N}{1-q}$

% Potenzreihe
\Bsp[Potenzreihe] Eine Potenzreihe kann man als eine Funktion 
\[
f(x)=\sum_{n=0}^\infty a_n
(x-x_0)^n
\]
auffassen. Es gilt:
\[
\begin{cases}
\abs{x-x_0}< \rho \quad \Longrightarrow \quad \sum_{n=0}^\infty a_nx^n \text{
konvergiert}\\
\abs{x-x_0}> \rho \quad \Longrightarrow \quad \sum_{n=0}^\infty a_nx^n \text{
divergiert}\\
\end{cases}
\]
Wobei je nach Eignung: 
\begin{align*}
\rho = \lim_{n\to\infty} \abs{\frac{a_n}{a_{n+1}}}, &\qquad n!, \ \alpha^{n}
\text{ oder Polynom}\\
\rho = \frac{1}{\lim_{n\to\infty}\sqrt[n]{\abs{a_n}}}. &\qquad(b_n)^n
\end{align*}
Beachte: Potenzreihen sind innerhalb ihres Konvergenzbereichs stetig.



% Zeta Funktion
\Bsp[Zeta-Funktion] Die Funktion konvergiert für $s>1$ und divergiert für $s=1$
$$\zeta(s)=\sum_{n=1}^{\infty} \frac{1}{n^{s}}$$

\sep
\begin{table}[H]
\centering
\begin{tabular}{p{4.5cm}p{1cm}}
$\lim_{n \rightarrow \infty} \sqrt[n]{n}$ & $=1$
\\
$\lim_{n \rightarrow \infty} n^{a}q^{n}, \ 0 \leq q \leq 1, \ a \in \Z$ & $=0$
\\
\midrule
$\lim _{n \rightarrow \pm \infty}\left(1 \pm \frac{x}{n}\right)^{n}$&$=e^{\pm x}$
\\
$\lim \limits_{n \rightarrow \infty \land f(n) \rightarrow \infty}\left(1+\frac{1}{f(n)}\right)^{f(n)}$ & $=e$
\\
$\lim _{x \rightarrow 0}(1+f(x))^{\frac{1}{f(x)}}$ & $=e$
\\
\midrule
$\lim _{x \rightarrow 0} \frac{\sin (x)}{x}$&$=1$
\end{tabular}
\end{table}